\chapter{2/6/2023}
Portfolio: $$ \mathbf P_t=\begin{pmatrix} P_{t,1} \\ P_{t,2} \\ \vdots \\ P_{t,q} \end{pmatrix} $$ and weights $\bs\alpha^\top=(\alpha_{t,1},\alpha_{t,2},\cdots,\alpha_{t,q})$. The initial value of the position is $$ \Pi_t=\bs\alpha_t^\top \mathbf P_t=\sum \alpha_{t,j}P_{t,j}. $$ A long position has $\alpha_{t,j}>0$ while a short position has $\alpha_{t,j}<0$. We define $$ w_{t,j}=\frac{\alpha_{t,j}P_{t,j}}{\Pi_t}. $$ Clearly, $\sum_{j=1}^q w_{t,j}=1$ for all $t$. The total long position is $$ \text{long}_t = \sum_{j=1}^q w_{t,j}\mathbf 1[w_{t,j}>0] $$ and similarly the total short position is $$ \text{short}_t = -\sum_{j=1}^q w_{t,j}\mathbf 1[w_{t,j}<0]. $$ By construction, $\text{long}_t-\text{short}_t=1$. Assuming no change in weights, the portfolio returns are then $$ r_{t+1}^\Pi=\frac{\Pi_{t+1}-\Pi_t}{\Pi_t}. $$ We can verify that this equals $$ \sum_{j=1}^q w_{t,j} r_{t+1,j}, $$ i.e. that the arithmetic returns of the portfolio is the weighted average of the returns of the individual substituents of the portfolio.

