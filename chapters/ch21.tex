\chapter{4/17/2023: More Stochastic Calculus, Option Pricing}
\section{Geometric Brownian Motion}
Recall that if $X_t = f(t,B_t)$, then \begin{align*}
	\mathrm dX_t &= \mathrm df(t,B_t) \\
				 &= \partial_t f(t,B_t)\mathrm dt + \frac{1}{2}\partial_{xx}^2 f(t,B_t)\mathrm dt + \underbrace{\partial_x f(t,B_t)\mathrm dB_t}_{\text{Martingale}}
\end{align*} We define Geometric Brownian motion using the differential equation $$ \mathrm dX_t = \mu X_t \mathrm dt + \sigma X_t \mathrm dB_t. $$ We can rearrange this to get \begin{align*}
	\frac{\mathrm dX_t}{X_t} &= \mu\mathrm dt + \sigma\mathrm dB_t \\
	\mathrm d\ln(X_t) &= \mu\mathrm dt + \sigma\mathrm dB_t.
\end{align*} Then, using It\^{o} and letting $X_t = f(t,B_t)$, we have \begin{align*}
	\mathrm dX_t &= \left( \partial_t f + \frac{1}{2}\partial_{xx}^2 f \right) \mathrm dt + \partial_x f\mathrm dB_t.
\end{align*} 
\subsection{Solution}
We will solve this differential equation by comparing and matching terms. First looking at the $\mathrm dB_t$ terms, \begin{align*}
	\partial_x f(t,B_t) &= \sigma X_t = \sigma f(t,B_t) 
	\implies f(t,x) = \exp \left\{ \sigma x + g(t) \right\}.
\end{align*} Next comparing the $\mathrm dt$ terms, we must have $$ \partial_t f + \partial_{xx}^2 f = \mu f, $$ which, using the explicit form we just found, we can manipulate to get $\partial_t f = f g'(t)$ and $\partial_{xx}^2 f = \sigma^2 f$, which then implies $$ \frac{\sigma^2}{2} + g'(t) = \mu \implies g(t) = \left(\mu-\frac{\sigma^2}{2}\right)t. $$ This gives the solution $$ X_t = X_0 \exp \left\{ \left( \mu-\frac{\sigma^2}{2} \right) t + \sigma B_t \right\}, $$ or $$ f(t,x) = \exp \left\{ \sigma x + \left( \mu-\frac{\sigma^2}{2} \right) t \right\} $$ assuming that $X_0 = 1$.

\subsection{Expected vs. Realized Properties}
We can further analyze this Geometric Brownian Motion object is $$ \mathbb E[X_t] = X_0 \mathbb E \left[ \exp\left\{\mu t+\sigma B_t - \frac{\sigma^2}{2}t\right\} \right] = X_0 e^{\mu t} \cdot \underbrace{\mathbb E \left[ \exp \left\{ \sigma B_t - \frac{\sigma^2}{2} t \right\}  \right]}_{1} = X_0 e^{\mu_t}. $$ The expression in the last expected value is called an exponential Martingale, and we will verify that its expected value is 1 in the next homework. 

Suppose $\mu=\frac{1}{2}$ and $\sigma=2$. As $t\to\infty$, we can see that $\mathbb E[X_t]\to\infty$, but looking at the actual individual values gives \begin{align*} X_t &= X_0\exp \left\{ \left( \mu-\frac{\sigma^2}{2} \right) t + \sigma B_t \right\} \\
		&= X_0 \exp \left\{ t \left[ \left( \mu-\frac{\sigma^2}{2} \right) + \frac{\sigma B_t}{t} \right]  \right\} .
\end{align*} Note that as $t\to\infty$, $\frac{\sigma B_t}{t}\to 0$ since it is distributed $\mathcal N \left( 0,\frac{\sigma^2}{t} \right)$. Thus, this quantity becomes $$ X_t\approx \exp \left\{ t \left( \mu-\frac{\sigma^2}{2} \right)  \right\} = \exp \left\{ -\frac{3}{2}t \right\} \overset{t\to\infty}{\longrightarrow} 0. $$ Even though the expected value is infinite, the actual values approach 0! This is because in the differential equation definition of $X_t$, the $\sigma$ term dominates the $\mu$ term, and thus the probability that it goes to some small value from which it never really recovers to become a large number increases towards 1 over time.

\section{Options Pricing Introduction}
We will start with a simple example. Suppose a stock is currently trading at 100 and $T$ will be worth either 120 (scenario $A$) or 90 (scenario $B$), each with probability $\frac{1}{2}$. How much would we be willing to pay for a call option with strike price 110 and expiration date $T=1$ given a current interest rate of 10\%?

\ex{Binomial Model}{Consider a portfolio with one short call and $\Delta$ units of long stock. Then under $A$ our portfolio is worth $\Delta\cdot 120 - 10$ and under B it is worth $\Delta\cdot 90$. If $\Delta=\frac{1}{3}$, then both of these values are equal to $30$, and thus is worth the same as a zero-coupon bond that pays $30$ at time $T$. Thus, for there to be no arbitrage, the amount of money we need to execute both of these strategies must be equal, so if $C$ is the price of the call option, then $$ C = \underbrace{\frac{100}{3}}_{\Delta\text{ stock}} - \underbrace{\frac{30}{1.1}}_{\substack{\text{Bond, adjusted} \\ \text{for Interest}}} = \frac{200}{33}\approx 6.06. $$ This is known as a hedging argument: we used a portfolio of a stock and option to synthetically replicate a bond. Most strikingly, the price of the option is independent of the probabilities of each case (presuming that they are positive).

For another slightly different (but essentially the same) solution, suppose we want to use a portfolio of stocks and bonds to replicate the payout of the option. If we have $\lambda$ stock and $\mu$ bonds, we solve the system of equations \begin{align*}
	120\lambda + \mu &= 10 \\
	90\lambda + \mu &= 0.
\end{align*} Solving this system gives us $\lambda=\frac{1}{3}$ and $\mu=-30$. This results in the same equation for $C$. This system of equations is known as the replication argument.}
