\chapter{4/5/2023: Options Continued}
\section{Options Review}
First we begin with an example showing the relationship between the different types of options contracts.
\ex{}{Suppose we take a long position in a stock that is currently worth $\$K$ and a short position in a call option for the same stock at strike price $K$ with premium $R$. Let $P$ be the price of the stock at the option's expiration. Then the payoff is $$ \underbrace{(P - K)}_{\text{profit of stock}} + \underbrace{R - \min(P-K,0)}_{\text{profit of option}} = \begin{cases} P-K+R & P\leq K \\ R & P>K \end{cases}. $$ Then the payoff graph will look the same as that of a short put, so we call this combination a synthetic short put.}

\section{Futures}
Before we delve into options pricing, we look into the pricing of futures, which are securities which are generally less complex than options. Now we switch to looking at futures/forward contracts.
\dfn{Futures/Forward Contracts}{An agreement to buy some amount of an asset at a future price on a specified future date.}
Suppose we have a future for 10000 ounces of gold on Christmas (time $T$) for price $K$. We can think of how to price this contract using an arbitrage argument. Suppose we borrow from the bank (which has 0 probability of defaulting) $Ke^{-r(T-t)}$ at time $t$ under the agreement that we will pay back to the bank $K$ at $T$ where $e^r$ is representing the risk free rate. Then consider the following two scenarios: we buy the asset now at the spot price of $S$, or we buy the future for the asset. Both of these should have the same present value. In the first case, we spend $S$. In the second case, we spend $S-X$ initially to buy the future for some value $X$ and then later buy the future for $X$. However, since we are spending $X$ less now, we can invest that amount into the risk free asset to get $Xe^{r(T-t)}$ at the expiration of the contract, so our value for the contract should include this interest. In order to have no arbitrage, the prices must be set so that $$ S = S - Xe^{r(T-t)} + K \implies X = K^{-r(T-t)}, $$ which gives that the price of the future should be $$ S - X = S - Ke^{-r(T-t)}. $$ 

\section{Back to Options}
Buying a call and selling a put at the same strike price $K$ and time of expiration $T$ is the same as buying a futures contract with the same expiration time $T$. If the price of the call is $C$ and the price of the put is $P$, then we must have that the price of the future equals $C-P$, so $$ C-P = S-K e^{-r(T-t)}. $$ This concept is known as \textbf{Put-Call Parity}.

\section{Brownian Motion}
Every realization of Brownian Motion will give a function that is continuous everywhere and differentiable nowhere. The intuition is based on scale invariance -- no matter how far you zoom in, it's still wiggly; you can't get any (arbitrarily good) linear approximations. If $x(t)=x_t)$, then if $\frac{\mathrm dx(u)}{\mathrm du} = a$, then $$\lim_{\delta u\to 0}\frac{x(u+\delta u)-x(u)}{\delta u} = a.$$ We can solve this equation by separating the variables: \begin{align*}
	\frac{\mathrm dx(u)}{\mathrm du} = a &\implies \mathrm dx(u) = a\mathrm dt \\
										 &\implies \int_0^t \mathrm dx(u) = a\int_0^t\mathrm du \\
										 &\implies x(t) - x(0) = at \\
										 &\implies x(t) = x_0 + at.
\end{align*} For another example, consider $\frac{\mathrm dx(t)}{\mathrm dt} = ax(t)$ given $x(0)=x_0$.

However, the market has a lot of stochasticity, so we want something like $\mathrm dx(t) = a\mathrm dt + z$, where $z\sim\mathcal N(0,\mathrm dt)$, but note that $z$ in this case is simply $\mathrm dB_t=B_{t+\delta t}-B_{t}$. However, this gets very tricky since Brownian Motion is differentiable nowhere.

