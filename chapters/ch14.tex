\chapter{3/22/2023: Guest Lecture -- Dmitry Malioutov}
Lots of stuff went on this lecture -- check the slides (if they are released).

\iffalse
Dmitry Malioutov works in portfolio management and applied machine learning at Millennium.

\section{Motivation}
Markowitz provided the pioneering insight into balancing the risk, reward, and benefits of diversification using a simple convex quadratic programming (QP) formulation.

\section{Classical Portfolio Theory: Review and Shortcomings}
We have a single-period optimization (e.g. 1 day, 1 month, etc.) from $t$ to $t+1$. We assume investors know (can estimate) expected future returns) and their covariance matrix (VCV matrix). This model ignores market frictions.

For the obejctive, we have reward as measured by expected returns and risk measured by variance (ignoring other risk metrics) -- our utility function is concave and balances risk and returns.

The price of a financial instrument at $t$ is $P_t$ and we define $R_t = R_{t\to t+1} = \frac{P_{t+1}-P_t}{P_t}$. If we have $N$ instruments, then we correspondingly have $\mathbf P_t = [P_{it}]$, $\mathbf R_t = [R_{it}]$. We also have portfolio weights $\mathbf w=[w_i]$ so that the portfolio returns are $R_{pt} = w^\top\mathbf R_t = \sum_{i=1}^N w_iR_{it}$. For portfolio construction we use simple returns, and for forecasting it's more convenient to use log-returns.

The expected future returns are $\bs\mu=\mathbb E[\mathbf R_t]$, and the covariance matrix of returns is $\bs\Sigma = [\Sigma_{ij}] = \Cov(R_{it},R_{ij})$. 
\dfn{Markowitz Portfolios Formulation, penalized version}{ $$\hat w(\gamma) = \max_{\gamma}\bs\mu^\top\mathbf w - \gamma\mathbf w^\top\bs\Sigma\mathbf w\quad\text{s.t.}\quad\mathbf w^\top\mathbf 1 = 1.$$} There are also other versions of Markowitz:
\dfn{Markowitz, constrained versions}{Target returns: $$ \hat w(\mu^\star) = \min_{\mathbf w}\mathbf w^\top\bs\Sigma\mathbf w\quad\text{s.t.}\quad\mathbf w^\top\bs\mu\geq\mu^\star,\mathbf 1^\top\mathbf w = 1 $$ Risk bound: $$ \hat w(S) = \max_{\mathbf w}\mathbf w^\top\bs\mu\quad{s.t.}\quad\mathbf w^\top\bs\Sigma\mathbf w\leq S, \mathbf 1^\top\mathbf w = 1 $$  }

\section{Statistical Estimation Challenges}

\section{Factor Models in Finance}

\section{Sparse Markowitz Portfolios}

\section{Market Frictions}

\section{Black-Litterman Portfolios}
\fi

